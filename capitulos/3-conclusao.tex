\chapter{CONSIDERAÇÕES FINAIS}
% ---

Em suma, a mobilidade entre as cidades do Vale do Jequitinhonha, especialmente no contexto do atual ciclo de extração de lítio, apresenta desafios significativos devido à escassez de veículos e à oferta limitada de serviços de transporte. A análise da frota de veículos e a comparação com o tamanho da população evidenciam a necessidade de soluções práticas para facilitar o deslocamento entre os municípios.

Além disso, a desigualdade na oferta de serviços em cada cidade intensifica as migrações pendulares, o que sobrecarrega as opções de transporte disponíveis atualmente. Nesse cenário, a implementação de um aplicativo de caronas pode ser uma solução inovadora, capaz de formalizar uma prática já existente, porém, agora, de forma mais segura e eficiente. Tal proposta visa não apenas garantir maior acessibilidade ao transporte, mas também promover preços justos e aumentar a oferta de viagens, contribuindo para o desenvolvimento socioeconômico da região

Durante o desenvolvimento do projeto foi possível criar vários \textit{endpoints} importantes para garantir um funcionamento inicial adequado, o que facilitará possíveis integrações com o \textit{frontend} futuramente. As funcionalidades de criação de conta, cadastro de motorista , autenticação de usuário e reserva de carona estão plenamente funcionais, além de seguirem o conceito do fundamental do \textit{CRUD (Create, Read, Update, Delete)}, essencial na criação de sistemas. 

%Por outro lado, algumas funções como a de verificação de documento, consulta veicular completa, pesquisa de antecedentes criminais e ranking de confiança não foram implementadas como o desejado, sendo adiadas para versões futuras da aplicação. Dessa forma, a API do “VemComigo” foi criada com sucesso e a sua existência é de suma importância para mitigar um dos efeitos notados como consequência das mudanças recentes no Vale do Jequitinhonha. Os testes de requisição foram satisfatórios e a segurança do sistema foi consolidada por meio de rotas protegidas e senhas criptografadas.



%Faça um breve resumo do tema e dos objetivos propostos para o trabalho. Em seguida, recapitule os principais resultados obtidos durante o desenvolvimento do sistema, apontando os marcos positivos e negativos que ocorreram durante o processo.

%Para finalizar o capítulo, descreva as limitações de sua solução e apresente propostas de melhorias futuras para o sistema.

%----


%Para ajudá-los a fazer citações de artigos, livros, sites e demais fontes de informação em LaTeX, deixarei esse pequeno parágrafo. Basicamente, há três formas de citar um material: 1) comando \textbf{cite}; 2) comando \textbf{citeonline}; e 3) o bloco \textbf{citacao}. 

%O comando \textbf{cite} apresenta o seguinte formato no PDF: \cite[p. 23]{abntex2-wiki-como-customizar}. O comando \textbf{cite} é usado quando se faz uma \textbf{citação direta curta}, após colocar-se o texto retirado da fonte entre as aspas.

%Já o comando \textbf{citeonline} apresenta o seguinte formato no PDF: \citeonline{abntex2-wiki-como-customizar}. Esse comando é usado quando se faz uma \textbf{citação indireta}, que é aquela onde se replica a \textbf{ideia do autor, mas as palavras são suas}. Esse tipo de formato de citação costuma aparecer ``entre o texto'', não no final.

%Por fim, o bloco citacao é utilizado para citações diretas que possuem mais de 3 linhas. Elas têm formatação diferente conforme as regras da \citeonline{NBR6024:2012}.

%\begin{citacao}
   % Esta é uma \textbf{citação direta longa}, ou seja, é uma citação que possui mais de 3 linhas copiadas literalmente de alguma fonte de informação utilizada no trabalho. Observe que o bloco \textbf{citacao} é usado em conjunto com o comando \textbf{cite}. \cite[p. 11]{abntex2-wiki-como-customizar}.
%\end{citacao}