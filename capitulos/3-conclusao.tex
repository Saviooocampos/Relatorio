\chapter{CONSIDERAÇÕES FINAIS}
% ---

Em suma, a mobilidade entre as cidades do Vale do Jequitinhonha, especialmente no contexto do atual ciclo de extração de lítio, apresenta desafios significativos devido à escassez de veículos e à oferta limitada de serviços de transporte. A análise da frota de veículos e a comparação com o tamanho da população evidenciam a necessidade de soluções práticas para facilitar o deslocamento entre os municípios.

Além disso, a desigualdade na oferta de serviços em cada cidade intensifica as migrações pendulares, o que sobrecarrega as opções de transporte disponíveis atualmente. Nesse cenário, a implementação de um aplicativo de caronas pode ser uma solução inovadora, capaz de formalizar uma prática já existente, porém, agora, de forma mais segura e eficiente. Tal proposta visa não apenas garantir maior acessibilidade ao transporte, mas também promover preços justos e aumentar a oferta de viagens, contribuindo para o desenvolvimento socioeconômico da região

Durante o desenvolvimento do projeto foi possível criar vários \textit{endpoints} importantes para garantir um funcionamento inicial adequado, o que facilitará possíveis integrações com o \textit{frontend} futuramente. As funcionalidades de criação de conta, cadastro de motorista , autenticação de usuário e reserva de carona estão plenamente funcionais, além de seguirem o conceito do fundamental do \textit{Create, Read, Update, Delete (CRUD)}, essencial na criação de sistemas. 
