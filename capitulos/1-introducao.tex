\chapter{INTRODUÇÃO}

A mineração é uma atividade imprescindível para a manutenção e o desenvolvimento da sociedade. O ser humano está sempre buscando novas formas de facilitar a sua sobrevivência por meio de ferramentas criadas com minérios extraídos da terra, que vão desde um simples livro, que necessita da celulose, um polímero extraído de fontes vegetais, até grandes invenções como aviões e computadores.

Nota-se um aumento na demanda mundial por lítio, causado por uma corrida incessante para a substituição da atual matriz energética. No Brasil, uma região se destaca na oferta de jazidas minerais: o Vale do Jequitinhonha.

Paralelamente ao progresso, a mineração de lítio, no entanto, levanta questões complexas que envolvem aspectos ambientais, econômicos e sociais. Este capítulo tem a finalidade de apresentar um relatório técnico sobre o desenvolvimento de um software de caronas, criado com o objetivo de resolver uma das demandas atuais causadas pelo novo ciclo econômico da região.



\section{Contextualização}

O “Lithium Valley Brazil” (Vale do Lítio Brasileiro) foi o nome dado ao novo projeto de extrativismo do Estado de Minas Gerais. No dia 9 de maio de 2023, durante um evento da bolsa de valores em Nova Iorque, conhecida como Nasdaq (National Association of Securities Dealers Automated Quotations; em português, "Associação Nacional de Corretores de Títulos de Cotações Automáticas"), o até então governador Romeu Zema liderou uma iniciativa responsável por atrair investidores do mundo inteiro. 

Mais uma vez, o Estado de Minas surpreendeu a indústria mundial com uma recente descoberta de ricas jazidas de lítio, mineral de suma importância para a economia global, sendo utilizado em ligas metálicas, medicamentos e, principalmente, nas baterias de celulares, computadores e carros elétricos. O lítio é extraído com a finalidade de ser exportado, assim como a maioria dos minérios.

Em julho de 2023, as primeiras 15 mil toneladas de lítio extraídas no Vale do Jequitinhonha foram entregues no Porto de Vitória, no Estado do Espírito Santo, assim sendo o pontapé inicial de um projeto audaz, coordenado pelo Governo de Minas Gerais, com a finalidade de atrair investimentos e empregos ao passo que promete desenvolver a região. A cobiça pelo “Ouro Branco”, nome popular do mineral, está associada a uma demanda cada vez maior por fontes de energia limpa como alternativa aos combustíveis fósseis. O lítio chega a ser classificado como “O Novo Petróleo”, assim afirmou Elon Musk, diretor geral da Tesla, empresa norte-americana de carros elétricos, em suas redes sociais.

Entre as empresas de mineração que operam na região em destaque encontra-se a Sigma Lithium, empresa canadense que se destaca no cenário global de extração do lítio. No início de 2023, a empresa inaugurou o seu complexo, atualmente o quarto maior produtor mundial, um projeto de extração baseado na sustentabilidade. Toda a cadeia de produção não usará barragens de rejeito, água potável, agentes químicos nocivos ao ambiente ou carvão mineral como fonte de energia, assim batizando o seu produto final como Lítio Verde.

Entretanto, apesar das expectativas criadas ao redor de tal minério, nota-se alguns impactos na região brasileira mais promissora para a extração do lítio. Uma das demandas causadas pelo atual ciclo econômico surge do fato de que a região carece de estrutura urbana adequada. Em uma reportagem do Brasil de Fato MG, moradores apontam problemas como superlotação de equipamentos públicos de saúde, adoecimento mental e físico, contaminação das águas, danos nas estruturas das casas e desgastes na malha rodoviária local. Vale ressaltar que as reservas estão localizadas no Norte e Nordeste de Minas Gerais, em uma região onde muitas cidades possuem baixos níveis no Índice de Desenvolvimento Humano (IDH).

Segundo a Secretaria de Desenvolvimento Econômico de Minas Gerais, o projeto Vale do Lítio é formado por 14 cidades: Araçuaí, Capelinha, Coronel Murta, Itaobim, Itinga, Malacacheta, Medina, Minas Novas, Pedra Azul, Virgem da Lapa, Teófilo Otoni e Turmalina, no Nordeste de Minas, e Rubelita e Salinas, no Norte mineiro. A instalação da mineradora estrangeira na Grota do Cirilo forçou uma intensa mudança nas cidades supracitadas devido à carência de mão de obra especializada e à ausência de serviços básicos em algumas cidades.

A falta de especialização local foi responsável por causar ondas de migrações de trabalhadores, muitos dos quais são de lugares diversos, atraídos com a possibilidade de bons salários e oportunidades de promoção. Seja na rede de hotelaria ou na oferta de alugueis, a região sofre com a especulação de preços causada pela elevada demanda por moradia paralela à baixa oferta de imóveis em uma única cidade. A solução encontrada por alguns dos trabalhadores foi buscar acomodações nas cidades próximas de onde trabalham, consequentemente causando as chamadas migrações pendulares.

Paralelamente, a falta de estrutura urbana na região dificulta a vida dos moradores. O Vale do Jequitinhonha carece de uma rede de transporte entre as cidades devido ao escasso número de linhas rodoviárias, que muitas das vezes operam em horários específicos, uma vez ao dia. Outra alternativa para o deslocamento seria o transporte por meio de veículos particulares, mas tal possibilidade é limitada pelo fato de que muitos ainda não possuem veículo próprio.

Portanto, entende-se que os desafios para a transformação da região por meio da mineração será um processo árduo. Muitos dos problemas enfrentados são ocorrências antigas agravadas pelas mudanças abruptas. É mister que o poder público invista em projetos para concentrar a cadeia produtiva do lítio no país, investindo em infraestrutura nas cidades em evidência. O objetivo deste trabalho é oferecer uma solução por meio do desenvolvimento de um aplicativo para mitigar o problema de deslocamento na região por meio de caronas.


\section{Objetivos}

O deslocamento entre as regiões é fundamental para trabalhadores da mineração e moradores das localidades. Uma característica do atual ciclo de extração do lítio no Vale do Jequitinhonha é o aumento no fluxo de movimentações entre as principais cidades, como Araçuaí, Itinga, Coronel Murta, Virgem da Lapa e Itaobim. Entretanto, um simples deslocamento pode se tornar difícil em algumas situações.

Primordialmente foi feita uma análise da oferta de veículos da região. Segundo dados do Ministério dos Transportes, SENATRAN - Secretaria Nacional de Trânsito - em 2023 o município de Araçuaí, para efeito de comparação, possui uma frota de 15667 veículos no total, entre os quais 4436 de passeio deles são automóveis. Paralelamente, o último censo do \citeonline{ibgePopulacao2022} apontou uma população de 34.297 pessoas, indicando que a região possui uma baixa oferta de veículos de passeio em comparação com o tamanho da população. Tal fato se repete nas demais regiões supracitadas.

Outro fator que evidencia a necessidade de migrações pendulares está no fato de que alguns municípios possuem serviços que os outros não oferecem. Historicamente, o desenvolvimento neles ocorreu de forma distinta. Antes do ciclo de mineração do lítio, era comum que os moradores se deslocassem de cidades ou povoados em busca de tratamento médico ou para realizarem compras nos centros comerciais de outras cidades. Muitos viajam entre as cidades por meio de táxi, carona ou por meio das poucas linhas de ônibus que operam em horários específicos. Atualmente, devido ao aumento da demanda de deslocamento, é comum que o acesso aos meios de transporte seja mais difícil. 

Dessa forma, este trabalho tem a finalidade de documentar a criação de uma possível solução para a demanda de transportes por meio de caronas. Um aplicativo destinado a isso pode aumentar a segurança de uma prática que já ocorre, porém de forma informal. Garantir preços justos e uma maior oferta de horários para viajar são objetivos do projeto.

\section{Público-alvo e Benefícios}


A carona é praticada na sociedade desde que os primeiros meios de transporte surgiram, por meio de cavalos e charretes. Normalmente, a carona é solicitada em ruas e estradas. No caso das estradas, utiliza-se um gesto universal: estender uma das mãos à frente do corpo com o polegar apontando na direção desejada. No entanto, um problema dessa prática é a falta de confiança entre passageiro e motorista.

Primordialmente, uma das finalidades do software é reduzir possíveis acontecimentos que prejudiquem a segurança dos envolvidos por meio da verificação do perfil de quem solicita a carona e de quem oferece a mesma, uma vez que a aplicação destina-se a todos que deslocam constantemente entre as cidades envolvidas no complexo de mineração do lítio. Alia-se a isso, a possibilidade de avaliar o perfil dos usuários conforme ocorre as viagens com uma nota e descrição. 

Outro benefício do programa seria o valor final de uma corrida. Um proprietário de um carro que viaja constantemente entre os municípios poderia oferecer uma carona como forma de reduzir as despesas com combustíveis, ao passo que uma pessoa que busca a carona poderia conseguir o transporte em um valor mais justo do que outros meios de transporte comuns na região.Vale citar que o valor dos deslocamentos intermunicipais estão sofrendo constantes reajustes no Estado de Minas Gerais, o que motiva as pessoas a utilizarem meios alternativos. 

Em uma reportagem do Divinews, jornal local da cidade de Divinópolis, fica evidente que o recente reajuste de 8\%, realizado em setembro de 2024, no valor das passagens intermunicipais fez com que os passageiros ficassem descontentes  com as empresas de viagens tradicionais. A notícia também relata que uma parte dos revoltados com a nova tributação não se importam de utilizar aplicativos de viagem, como o Buser, ou de aceitar caronas oferecidas em grupos de Facebook ou Whatsapp.

Caso o motorista tenha interesse e disponibilidade de espaço no seu veículo, ele poderá oferecer uma carona gratuita. A finalidade de ofertar tal recurso de forma não remunerada é preencher as vagas ociosas de seus carros em uma viagem que já ocorreria normalmente, ao passo que o motorista poderia ser beneficiado com uma companhia durante todo o trajeto.




%\section{Plano de monetização}

%Descreva como a equipe planeja monetizar o produto elaborado (sistema construído). Quais os eventuais valores, formas de cobrança e itens relacionados ao negócio. Para isso, suponha o 
